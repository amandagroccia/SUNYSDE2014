\numberwithin{equation}{section}

\subsection{Histograms}
%\begin{frame}
  %\frametitle{Outline}
  %\tableofcontents[ currentsection ]
%\end{frame}

\begin{frame}{Description of Simulations}

  Describe what was done for the following two sets of histograms.

\end{frame}


\begin{frame}{Results}
 

  \begin{columns}[t]
    \column{.5\textwidth} 
    %\includegraphics[width=6cm]{img/Heun500}
    histogram from Matlab simulations
    \column{.5\textwidth}
    %\includegraphics[width=6cm]{img/Heun10000}
    histogram from C simulations
  \end{columns}


\end{frame}



\subsection{Approximations}

\begin{frame}{Milstein Method}

    \textbf{This is not correct.}

	\begin{itemize}
		\item Technique for the approximate numerical solution of a stochastic differential equation
		\item Increases accuracy by using It\^o's Formula by adding the second-order term
	\end{itemize} 

	\begin{align*}
		Y_{i+1} = Y_i +b\Delta t + \sigma \Delta W_t + \frac{1}{2} \sigma \sigma_x \{(\Delta W_t)^2 - \Delta t\} 
	\end{align*}

\end{frame}

\begin{frame}{Full Approximations}
RESULTS \\
Because this is a random variable each time it's run it's different. \\
MOVIE INSERT 
\end{frame}

\begin{frame}{Confidence Intervals}

  \textbf{I have no idea what this means or what was done???}

  In order to better analyze our data we want to calculate our
  confidence intervals. To do this we needed to come up with a formula to calculate them. The following formula is what was used to find our confidence intervals. \\

  $$\hat{\theta} - S(\hat{\theta}) L \leq \theta \leq \hat{\theta} + S(\hat{\theta}) L$$ \\

		Where $\hat{\theta}$ is the probability of each frequency table, \\
		$S(\hat{\theta})$ is ... \textbf{I don't know how you want to put this into ``math terms''}, \\ 
		$L$ is \textbf{I don't know how you want to put this into ``math terms''}.
\end{frame}

\begin{frame}{Experimental Design}
  We need to estimate the number of trials. To do this we solved the following formula for $n_1$, the number of species in each frequency table, and multiplied by the number of frequency tables. 
	
\vfill

By letting $P_1 = \frac{1}{2}$ and solving for $L = 5.991465$ in R, we get:

  $$\sqrt{\frac{P_1 (1-P_1)}{n_1}} \ L = .01$$ \\
  $$n_1 = \frac{L^2 P_1 (1-P_1)}{.0001}$$ \\
  After multiplying by $3$ (our number of frequency tables) we get: \\
  \begin{center} 
    $N = 44935.9875$ 
  \end{center} 

  From this we knew to run our programs $45,000$ times. 
	
\vfill

\end{frame}

\begin{frame}{Frequency Tables}

		Our frequency tables are divided into three groups:
\begin{itemize}
	\item species 1 lives while species 2 dies out
	\item species 2 lives while species 1 dies out
	\item species 1 and species 2 lives
\end{itemize}
\end{frame}


